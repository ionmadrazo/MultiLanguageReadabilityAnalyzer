% This is "sig-alternate.tex" V2.1 April 2013
% This file should be compiled with V2.5 of "sig-alternate.cls" May 2012
%
% This example file demonstrates the use of the 'sig-alternate.cls'
% V2.5 LaTeX2e document class file. It is for those submitting
% articles to ACM Conference Proceedings WHO DO NOT WISH TO
% STRICTLY ADHERE TO THE SIGS (PUBS-BOARD-ENDORSED) STYLE.
% The 'sig-alternate.cls' file will produce a similar-looking,
% albeit, 'tighter' paper resulting in, invariably, fewer pages.
%
% ----------------------------------------------------------------------------------------------------------------
% This .tex file (and associated .cls V2.5) produces:
%       1) The Permission Statement
%       2) The Conference (location) Info information
%       3) The Copyright Line with ACM data
%       4) NO page numbers
%
% as against the acm_proc_article-sp.cls file which
% DOES NOT produce 1) thru' 3) above.
%
% Using 'sig-alternate.cls' you have control, however, from within
% the source .tex file, over both the CopyrightYear
% (defaulted to 200X) and the ACM Copyright Data
% (defaulted to X-XXXXX-XX-X/XX/XX).
% e.g.
% \CopyrightYear{2007} will cause 2007 to appear in the copyright line.
% \crdata{0-12345-67-8/90/12} will cause 0-12345-67-8/90/12 to appear in the copyright line.
%
% ---------------------------------------------------------------------------------------------------------------
% This .tex source is an example which *does* use
% the .bib file (from which the .bbl file % is produced).
% REMEMBER HOWEVER: After having produced the .bbl file,
% and prior to final submission, you *NEED* to 'insert'
% your .bbl file into your source .tex file so as to provide
% ONE 'self-contained' source file.
%
% ================= IF YOU HAVE QUESTIONS =======================
% Questions regarding the SIGS styles, SIGS policies and
% procedures, Conferences etc. should be sent to
% Adrienne Griscti (griscti@acm.org)
%
% Technical questions _only_ to
% Gerald Murray (murray@hq.acm.org)
% ===============================================================
%
% For tracking purposes - this is V2.0 - May 2012

\documentclass{sig-alternate-05-2015}
\usepackage{color}

\begin{document}

% Copyright
\setcopyright{acmcopyright}
%\setcopyright{acmlicensed}
%\setcopyright{rightsretained}
%\setcopyright{usgov}
%\setcopyright{usgovmixed}
%\setcopyright{cagov}
%\setcopyright{cagovmixed}


% DOI
\doi{10.475/123_4} 

% ISBN
\isbn{123-4567-24-567/08/06}

%Conference
\conferenceinfo{PLDI '13}{June 16--19, 2013, Seattle, WA, USA}

\acmPrice{\$15.00}

%
% --- Author Metadata here ---
\conferenceinfo{WOODSTOCK}{'97 El Paso, Texas USA}
%\CopyrightYear{2007} % Allows default copyright year (20XX) to be over-ridden - IF NEED BE.
%\crdata{0-12345-67-8/90/01}  % Allows default copyright data (0-89791-88-6/97/05) to be over-ridden - IF NEED BE.
% --- End of Author Metadata ---

\title{YouUnderstood.me? Readability based retrieval of reading materials for students and educators}

%
% You need the command \numberofauthors to handle the 'placement
% and alignment' of the authors beneath the title.
%
% For aesthetic reasons, we recommend 'three authors at a time'
% i.e. three 'name/affiliation blocks' be placed beneath the title.
%
% NOTE: You are NOT restricted in how many 'rows' of
% "name/affiliations" may appear. We just ask that you restrict
% the number of 'columns' to three.
%
% Because of the available 'opening page real-estate'
% we ask you to refrain from putting more than six authors
% (two rows with three columns) beneath the article title.
% More than six makes the first-page appear very cluttered indeed.
%
% Use the \alignauthor commands to handle the names
% and affiliations for an 'aesthetic maximum' of six authors.
% Add names, affiliations, addresses for
% the seventh etc. author(s) as the argument for the
% \additionalauthors command.
% These 'additional authors' will be output/set for you
% without further effort on your part as the last section in
% the body of your article BEFORE References or any Appendices.

\numberofauthors{2} %  in this sample file, there are a *total*
% of EIGHT authors. SIX appear on the 'first-page' (for formatting
% reasons) and the remaining two appear in the \additionalauthors section.
%
\author{
% You can go ahead and credit any number of authors here,
% e.g. one 'row of three' or two rows (consisting of one row of three
% and a second row of one, two or three).
%
% The command \alignauthor (no curly braces needed) should
% precede each author name, affiliation/snail-mail address and
% e-mail address. Additionally, tag each line of
% affiliation/address with \affaddr, and tag the
% e-mail address with \email.
%
% 1st. author
\alignauthor
Ion Madrazo\\
       \affaddr{Computer Science Department}\\
       \affaddr{Boise State University}\\
       \affaddr{Boise, Idaho, USA}\\
       \email{ionmadrazo@boisestate.edu}
% 2nd. author
\alignauthor
Sole Pera\\
       \affaddr{Computer Science Department}\\
       \affaddr{Boise State University}\\
       \affaddr{Boise, Idaho, USA}\\
       \email{solepera@boisestate.edu}
}
% There's nothing stopping you putting the seventh, eighth, etc.
% author on the opening page (as the 'third row') but we ask,
% for aesthetic reasons that you place these 'additional authors'
% in the \additional authors block, viz.

\date{30 July 1999}
% Just remember to make sure that the TOTAL number of authors
% is the number that will appear on the first page PLUS the
% number that will appear in the \additionalauthors section.

\maketitle
\begin{abstract}


K-12 students make use of online resources to fulfill their academic information needs on a daily basis. However, they can often get discouraged because the contents they retrieve are outside their comprehension level, whether being too easy or too difficult  for them to read. On the other side, educators find several challenges in finding materials for their classes that suit the students reading skills. Those reasons, make both students and educators spend a reasonably large  amount of time seeking for adequate materials.

 In this paper, we present a web application, that makes use of multidisciplinary techniques, such as, natural language processing, machine learning and information retrieval techniques, to help both students and educators in the process of finding materials that suit the reading skills of the students in a faster and more efficient way. For this purpose, the web application we present, puts together: (1) a material searcher that by combining a search engine and a readability formula, permits the fast retrieval of documents from different sources, (2)  a readability tracking system that enables both types of users to see how the reading skills of the student are evolving with time and (3) an analysis tool that enables educators to analyse materials from outside the application for determining their complexity level.






\end{abstract}


%
% The code below should be generated by the tool at
% http://dl.acm.org/ccs.cfm
% Please copy and paste the code instead of the example below. 
%



 \begin{CCSXML}
<ccs2012>
<concept>
<concept_id>10002951.10003317.10003331.10003271</concept_id>
<concept_desc>Information systems~Personalization</concept_desc>
<concept_significance>500</concept_significance>
</concept>
<concept>
<concept_id>10002951.10003317.10003331.10003336</concept_id>
<concept_desc>Information systems~Search interfaces</concept_desc>
<concept_significance>500</concept_significance>
</concept>
<concept>
<concept_id>10002951.10003317.10003347.10003356</concept_id>
<concept_desc>Information systems~Clustering and classification</concept_desc>
<concept_significance>300</concept_significance>
</concept>
<concept>
<concept_id>10010147.10010178.10010179</concept_id>
<concept_desc>Computing methodologies~Natural language processing</concept_desc>
<concept_significance>500</concept_significance>
</concept>
</ccs2012>
\end{CCSXML}

\ccsdesc[500]{Information systems~Personalization}
\ccsdesc[500]{Information systems~Search interfaces}
\ccsdesc[300]{Information systems~Clustering and classification}
\ccsdesc[500]{Computing methodologies~Natural language processing}





%
% End generated code
%

%
%  Use this command to print the description
%
\printccsdesc

% We no longer use \terms command
%\terms{Theory}

\keywords{Search engines; Filtering; Readability assessment; Student tracking}

\section{Introduction}

K-12 students make use of the internet in a daily basis to fulfill their information needs for completing their academic tasks, such as, finding information for a class presentation or discovering the meaning of a new word. For this purpose, they mostly use search engines and online available catalogues for retrieving contents such as, news articles, books or term definitions. However, they can often get discouraged because the contents they retrieve are outside their comprehension level, whether being too easy or too difficult  for them to read.\\

%Reading is an important skill in the academic environment, a skill that can have a high impact for a student's educational opportunities and their career \cite{robinson2000issues}. However, students can get discouraged to read, when the texts provided are outside of their comprehension level, whether being too easy to read  or too difficult for them.


In the academic environment students are not the only ones facing the problem of retrieving adequate contents in terms of reading abilities and information needs. For example, even in a same grade class, students' reading skills can differ significantly, so not all students in the same class can be provided with same texts. This supposes a personalization need that the instructor needs to handle on a daily basis. However with the high number of students in class this task becomes impossible to tackle. All of these reasons, make instructors spend a significant amount of their time seeking adequate materials for their students, a time that could be reduced with the help of an specialized application.\\


We have developed an application oriented to both instructors and students in the process of finding materials that suit the students reading level. The system is centered on a readability formula that together with a search engine makes looking for leveled reading material easier and more efficient. The system lets students log in in the web application, which keeps track of the materials read by each student and his feed-back for each or the read materials(too easy/OK/too complex). This enables the application to make suggestions about the readability level for each student. Both the student and the teacher are able too see the readability score in order to see how it evolves. Furthermore, the application users can use the included material searcher which retrieves results tailored to the readability score. The material searcher can act as an interface for various types of material sources depending on the capabilities of the education center that makes use of the application. Internal sources such as, school libraries or external resources such as online book catalogues like AR\footnote{http://www.acceleratelearning.com} or Lexile\footnote{http://lexile.com} or encyclopaedias such as Wikipedia.Moreover, even the internet can act as a source for the application, enabling it to be used as a conventional search engine, which filters results depending on the readability score. {\color{red} Cannot find a way of structuring this last part, beginning at "The material searcher..."}





Apart from the search engine, the instructors have access to an analysis page, where they can submit texts they found outside the application for determining their readability score. This tool, together with the track of readability scores of each students, helps them make sure the materials they found are adequate or not for the class.

,


\section{Readability assessment}
Whether when a student is searching information for completing an assignment or when an educator uses the application for finding material for his course, a text's complexity needs to be determined.\\

% \textbf{YouUnderstand.me} enables the use of different tools for that purpose.\\


Different approaches have been followed in the literature for determining a text's complexity or readability. Most approaches, orient their work to determining the readability of a general text. Those systems vary, from the very simple ones[Ref], which make use of shallow features, such as, the average number of words per sentence or the average length of terms, to more complex ones [Ref], which are mostly based, on supervised learning techniques and features extracted using Natural language processing. However, those tools have shown to be of small use in contexts where the text of the reading material has reduced accessibility, both because the texts is not publicly accessible or because it shows a structure that is not as simple to tackle . Therefore, different works have been done in more specialized contexts such as book[Ref] or web page retrieval[Ref], where the systems presented made use other features apart from the ones in the text.\\


\textbf{YouUnderstood.me} can make use of different readability assessment metrics at the same time, aiming to be able to handle a more diverse amount of reading materials. 

%However, our application needs to be more general in order to be able to adapt to the different needs of the education centres. Therefore, \textbf{YouUnderstand.me}



%Lead-in Sea estudiante o profesor, lo que importa es medirle el texto usando algun sistema de classficiation standarizado. Hay escuelas que utilizan AR o lexile, pero eso solo sirve con esas herramientas. Existen herramientas apra solo libros y tambien apra solo web, pero encesitamos algo independiente. Algo simpre --> algo complejo, nuestor sistema.


%Hay tools para books TROLL otro apra webs, el nuestro es independiente
%Hay escuelas que usas ciertos sistemas de lectura AR Lexile
%Sistema de lectura simples

Most of the features of the application, from material retrieval filtering to external material analysis, are based on the complexity analysis of the reading materials. This analysis is performed by the {\color{red}XXXXX} \cite{imadrazo2016readability} readability assessment tool, which,  is capable of detecting the input language of a text on the fly and providing a readability score for it.\\

{\color{red}XXXXX} is based on a supervised learning paradigm, at the moment, makes use of more than a hundred features for learning and prediction. The features are extracted using Natural Language Processing tools, such as a tokenizer, a part of speech tagger and different semantic analyzers. Those features, are used to train a model that will later be used for predicting the readability score for new materials.



%The document analyzer is the center of the the application
%It is used for both analysis tool and material searcher
%




\section{Material searcher}


\section{Tracking students}

\section{User Experience}

\begin{itemize}
\item \textbf{Analyzing materials.} An analysis page is provided to instructors, so that they can make use of the readability assessment algorithm \cite{imadrazo2016readability} with materials from outside the application. This page provides an input form where the instructor can submit the material, and receive a readability level prediction for the it.
\item \textbf{Looking for new materials.} A material search page is provided so that the user can insert text queries for looking for materials. The user can select a certain readability level he wants to look for, of leave it blank letting the application choose the best level for the logger user. Furthermore, the material source can be selected, depending on the user needs. In case of students, they can select a material for reading and provide feed-back on it. This feed-back will be used for tracking purposes.


\item \textbf{Tracking students.} The educator can see a table with data about all his students, where the number of materials read and the readability score for each student are provided. The instructor can go deeper if needed, and see each of the materials each students has read and the individual readability scores for each material, as well as, the feed-back given by the student. The data is also presented in a summarized way, so that the educator can see the average, maximum and minimum reading skills of the students in class.

\end{itemize}


\section{Future work}

{\color{red}
\section{Missing ideas}
\begin{itemize}
\item The application is fully multilingual.
\end{itemize}

\begin{comment}
\section{Title ideas}
\begin{itemize}
\item YouUnderstood.me?
\item YouUnderstood.me? Academic material retrieval for students and educators
\item YouUnderstood.me? A readability filtered material searcher for students and educators
\item  YouUnderstood.me? Making sure student s and educators understand each other
\item  YouUnderstood.me? Making students and educators understand each other
\end{itemize}
\end{comment}
}


\bibliography{bibliography}{}
\bibliographystyle{apalike}
\nocite{*}
\end{document}
